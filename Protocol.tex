%! TeX program = lualatex

%%%%%%%%%%%%%%%%%%%%%%%%%%%%%%%%%%%%%%%%%%%%%%%%%%%%%%%%%%%%%%
% 1. Document Configuration Settings
%%%%%%%%%%%%%%%%%%%%%%%%%%%%%%%%%%%%%%%%%%%%%%%%%%%%%%%%%%%%%%

% Set document type to article with a default font size of 12pt
\documentclass[12pt]{article}

%%%%%%%%%%%%%%%%%%%%%%%%%%%%%%%%%%%%%%%%%%%%%%%%%%%%%%%%%%%%%%
% 2. Font and Typography Settings
%%%%%%%%%%%%%%%%%%%%%%%%%%%%%%%%%%%%%%%%%%%%%%%%%%%%%%%%%%%%%%

% Paragraph spacing and indentation
\usepackage{parskip} % Adds space between paragraphs
\setlength{\parindent}{0pt} % No paragraph indentation
\setlength{\parskip}{6pt plus 2pt minus 1pt} % Set paragraph spacing

%%%%%%%%%%%%%%%%%%%%%%%%%%%%%%%%%%%%%%%%%%%%%%%%%%%%%%%%%%%%%%
% 3. Math Support
%%%%%%%%%%%%%%%%%%%%%%%%%%%%%%%%%%%%%%%%%%%%%%%%%%%%%%%%%%%%%%

% Math support packages
\usepackage{amsmath} % For math support
\usepackage{amssymb} % For additional math symbols
\usepackage{unicode-math} % For Unicode math

%%%%%%%%%%%%%%%%%%%%%%%%%%%%%%%%%%%%%%%%%%%%%%%%%%%%%%%%%%%%%%
% 4. Geometry and Figure Handling
%%%%%%%%%%%%%%%%%%%%%%%%%%%%%%%%%%%%%%%%%%%%%%%%%%%%%%%%%%%%%%

% Set page margins to 1 inch
\usepackage[margin=1in]{geometry}

% Include graphics
\usepackage{graphicx}
\def\maxwidth{\ifdim\Gin@nat@width>\linewidth\linewidth\else\Gin@nat@width\fi} % Adjust image width
\def\maxheight{\ifdim\Gin@nat@height>\textheight\textheight\else\Gin@nat@height\fi} % Adjust image height

%%%%%%%%%%%%%%%%%%%%%%%%%%%%%%%%%%%%%%%%%%%%%%%%%%%%%%%%%%%%%%
% 5. Tables and Figures Customization
%%%%%%%%%%%%%%%%%%%%%%%%%%%%%%%%%%%%%%%%%%%%%%%%%%%%%%%%%%%%%%

% Table packages
\usepackage{booktabs} % High-quality tables
\usepackage{longtable} % Long tables spanning pages
\usepackage{array} % Advanced table features
\usepackage{multirow} % Merge rows in tables
\usepackage{colortbl} % Table color customization

% Customize table cells
\usepackage{makecell}

% Caption customization
\usepackage{caption}

% TikZ for diagrams
\usepackage{tikz}

%%%%%%%%%%%%%%%%%%%%%%%%%%%%%%%%%%%%%%%%%%%%%%%%%%%%%%%%%%%%%%
% 6. List Customization
%%%%%%%%%%%%%%%%%%%%%%%%%%%%%%%%%%%%%%%%%%%%%%%%%%%%%%%%%%%%%%

% Custom list environment
\usepackage{enumitem}

%%%%%%%%%%%%%%%%%%%%%%%%%%%%%%%%%%%%%%%%%%%%%%%%%%%%%%%%%%%%%%
% 7. Bibliography and Citation Settings
%%%%%%%%%%%%%%%%%%%%%%%%%%%%%%%%%%%%%%%%%%%%%%%%%%%%%%%%%%%%%%

% BibLaTeX for bibliography management
\usepackage[backend=biber, style=numeric, sorting=none]{biblatex}

% Add bibliography file
\addbibresource{Protocol.bib}

%%%%%%%%%%%%%%%%%%%%%%%%%%%%%%%%%%%%%%%%%%%%%%%%%%%%%%%%%%%%%%
% 8. Hyperlinks and URL Handling
%%%%%%%%%%%%%%%%%%%%%%%%%%%%%%%%%%%%%%%%%%%%%%%%%%%%%%%%%%%%%%

% Hyperlinks
\usepackage{hyperref}
\hypersetup{hidelinks, pdfcreator={LaTeX via pandoc}} % Hide hyperlink boxes

% Set URL style to match text
\urlstyle{same}

%%%%%%%%%%%%%%%%%%%%%%%%%%%%%%%%%%%%%%%%%%%%%%%%%%%%%%%%%%%%%%
% 9. Title, Subtitle, and Author Setup
%%%%%%%%%%%%%%%%%%%%%%%%%%%%%%%%%%%%%%%%%%%%%%%%%%%%%%%%%%%%%%

% Title of the document
\title{RESEARCH PROTOCOL\\Study Repo Template}

% Subtitle of the document
\usepackage{etoolbox}
\makeatletter
\providecommand{\subtitle}[1]{% Add subtitle to title
  \apptocmd{\@title}{\par {\large #1 \par}}{}{}
}
\makeatother
\subtitle{Version: 0.0.1}

% Author setup 
\author{Jakub Mitura, Jacob S. Zelko}

% Suppress date with negative vertical space
\date{\vspace{-2.5em}}

%%%%%%%%%%%%%%%%%%%%%%%%%%%%%%%%%%%%%%%%%%%%%%%%%%%%%%%%%%%%%%
% 10. Fancy Header and Footer Setup
%%%%%%%%%%%%%%%%%%%%%%%%%%%%%%%%%%%%%%%%%%%%%%%%%%%%%%%%%%%%%%

% Fancy header/footer settings
\usepackage{fancyhdr}
\pagestyle{fancy}
\renewcommand{\headrulewidth}{0pt}
\setlength\headheight{30.0pt}
\addtolength{\textheight}{-30.0pt}

% Custom header with image
\chead{\includegraphics[width=\textwidth]{OHDSI_page_header}}
\lhead{}
\rhead{}

%%%%%%%%%%%%%%%%%%%%%%%%%%%%%%%%%%%%%%%%%%%%%%%%%%%%%%%%%%%%%%
% DOCUMENT BEGINS
%%%%%%%%%%%%%%%%%%%%%%%%%%%%%%%%%%%%%%%%%%%%%%%%%%%%%%%%%%%%%%

% Begin document
\begin{document}

% Title
\maketitle

% Table of Contents
\tableofcontents

\section{List of Abbreviations}
\label{list-of-abbreviations}

% Example of abbreviations table
\begin{table}[!h]
\centering\begingroup\fontsize{8}{10}\selectfont
\begin{tabular}{ll}
\toprule
\cellcolor{gray!10}{CDM} & \cellcolor{gray!10}{Common data model}\\
\bottomrule
\end{tabular}
\endgroup{}
\end{table}

\section{Abstract}
\label{abstract}

\textbf{Background and Significance}:\\
This section provides an overview of the importance and context of the research. The significance of studying the impact of interventions on health outcomes is highlighted here. \cite{Schuemie2018-zi}

\textbf{Study Aims}:\\
The main aim of this study is to evaluate the effectiveness of the intervention in reducing health risks.

\textbf{Study Description}:\\
This study will use a longitudinal design to assess the impact of the intervention over a period of six months. Data will be collected from various sources.

\begin{itemize}
  \item \textbf{Population}: Adult patients with chronic conditions.
  \item \textbf{Comparators}: Standard care versus intervention group.
\end{itemize}

\section{Amendments and Updates}
\label{amendments-and-updates}

This section will document any changes made to the research protocol after the initial approval. Updates to the study design or methodology will be noted here.

\section{Milestones}
\label{milestones}

Key milestones in the project timeline will be outlined in this section. These include major phases of the research, such as data collection, analysis, and reporting.

\section{Rationale and Background}
\label{rationale-and-background}

This section provides the background information and justification for the study. It includes a review of existing literature and the gaps this study aims to address.

\section{Study Objectives}
\label{study-objectives}

\subsection{Primary Hypotheses}
The primary hypothesis of this study is that the intervention will significantly reduce the incidence of adverse health events compared to standard care.

\subsection{Secondary Hypotheses}
Secondary hypotheses include the expectation that the intervention will improve overall quality of life and reduce healthcare costs.

\subsection{Primary Objectives}
The main objectives are to evaluate the effectiveness of the intervention in reducing health risks and to assess patient outcomes.

\subsection{Secondary Objectives}
Secondary objectives include examining changes in quality of life and cost-effectiveness of the intervention.

\section{Research Methods}
\label{research-methods}

\subsection{Study Design}
The study will employ a randomized controlled trial design to compare the intervention with standard care.

\subsection{Data Sources}
Data will be sourced from electronic health records, patient surveys, and clinical assessments.

\subsection{Study Population}
The study population will include adult patients with chronic conditions who meet the inclusion criteria.

\subsection{Exposures}
The primary exposure is the intervention being tested, with comparisons made to the standard care.

\subsection{Outcomes}
Outcomes include rates of adverse health events, quality of life measures, and healthcare utilization.

\subsection{Covariates}
Covariates will include age, gender, and comorbid conditions, which may influence the study outcomes.

\section{Data Analysis Plan}
\label{data-analysis-plan}

\subsection{Calculation of Time-at-Risk}
The time-at-risk will be calculated based on the duration of patient exposure to the intervention and follow-up periods.

\subsection{Model Specification}
Statistical models will be specified to account for the effects of the intervention and other covariates.

\subsection{Pooling Effect Estimates Across Databases}
Effect estimates will be pooled from multiple databases to increase statistical power and generalizability.

\subsection{Analyses to Perform}
Analyses will include comparisons of outcomes between intervention and control groups, as well as sensitivity analyses.

\subsection{Output}
The results will be presented in terms of effect sizes, confidence intervals, and statistical significance.

\section{Evidence Evaluation}
\label{evidence-evaluation}

\subsection{Study Diagnostics}
Diagnostics will include assessments of model fit and validation of the analysis methods.

\subsection{Sample Size and Study Power}
Sample size calculations and power analyses will ensure the study is adequately powered to detect significant effects.

\subsection{Cohort Comparability}
Comparability of the intervention and control cohorts will be assessed using baseline characteristics and propensity scores.

\subsection{Systematic Error Assessment}
Systematic errors and biases will be evaluated and addressed in the analysis.

\section{Strengths and Limitations of the Research Methods}
\label{strengths-and-limitations}

This section will discuss the strengths of the research design, such as its rigor and validity, as well as limitations, including potential sources of bias.

\section{Protection of Human Subjects}
\label{protection-of-human-subjects}

Details on how human subjects will be protected, including informed consent procedures and confidentiality measures, will be described.

\section{Management and Reporting of Adverse Events and Adverse Reactions}
\label{management-of-adverse-events}

Procedures for monitoring, documenting, and reporting adverse events and reactions will be outlined.

\section{Plans for Disseminating and Communicating Study Results}
\label{dissemination-of-results}

Plans for disseminating the study results through publications, presentations, and other channels will be described.

\section{Appendix: Negative Controls}
\label{appendix-negative-controls}

This appendix will include information on negative controls used in the study to test for non-specific effects.

\section{References}
\label{references}

\printbibliography

\end{document}
